\documentclass[journal]{IEEEtran}

\usepackage{blindtext}
\usepackage{graphicx}
\usepackage{cite}
\usepackage[utf8x]{inputenc}
\usepackage[slovene,english]{babel}  
\usepackage{url}

\hyphenation{op-tical net-works semi-conduc-tor}
\begin{document}
\title{IMapBooks - Automatic Question Response Grading}
\author{Tilen Tomakić\\tt5157@student.uni-lj.si}%
\maketitle

%  Abstract -  summary of problem and contribution
\begin{abstract}	
  Ovrednotenje pravilnosti odgovorov na zastavljena vprašanja v zvezi s podanim besedilom. Dokument opisuje primitivne postopke za točkovanje odgovorov.
\end{abstract}

\begin{IEEEkeywords}
NLP, IMapBook
\end{IEEEkeywords}

\IEEEpeerreviewmaketitle

% Introduction - problem motivation and background
\section{Uvod}
Naloga ovrednotenja pravilnosti odgovora oziroma bolj splošno izjave se lahko lotimo na več načinov. Metode pa nam v širšem

%  Related work - relevant literature overview
\section{Podobna dela}


%  Methods - used methods and techniques
\section{Uporabljene metode}
\subsection{Model A}
\subsection{Model B}
\subsection{Model C}

% Results - results description
\section{Rezultati}

%Discussion - results comparison and evaluation
\section{Discussion}

\section{Izvorna koda}
Izvorna koda je na voljo na git repozitoriju: https://github.com/TilenTomakic/IMapBooks-AQRG

\section{Zaključek}
oo~\cite{adhya2016automated}

\ifCLASSOPTIONcaptionsoff
  \newpage
\fi

% literature (use BibTeX)
\bibliographystyle{IEEEtran}
\bibliography{literatura}

\end{document}
